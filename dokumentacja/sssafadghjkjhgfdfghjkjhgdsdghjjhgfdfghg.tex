\documentclass[12pt,a4paper]{article}
\usepackage[latin2]{inputenc}
\usepackage[T1]{fontenc}
\usepackage{graphicx}	% do impotu grafik
\usepackage{amsthm}
\usepackage{amsfonts}
\usepackage{xcolor}

\usepackage{listings} %do impotu kodu
\author{Marcin J?drzejczyk \and Sebastian Katszer }
\title{Projekt UML Automatu z przek?skami }


\newcommand{\HRule}{\rule{\linewidth}{0.5mm}} % Defines a new command for the horizontal lines, change thickness here
\newtheorem{defi}{Definicja}
\renewcommand{\contentsname}{Spis tre?ci}
\renewcommand{\refname}{Bibliografia}
\renewcommand{\figurename}{Rysunek}


\begin{document}

%\maketitle	%robi stron� tytu�ow�
\begin{titlepage}


\center % Center everything on the page
%----------------------------------------------------------------------------------------
%	HEADING SECTIONS
%----------------------------------------------------------------------------------------

\textsc{\LARGE Akademia G�rniczo-Hutnicza    \\im. Stanis?awa Staszica w Krakowie}\\[1.5cm] % Name of your university/college
\textsc{\Large Analiza i modelowanie oprogramowania}\\[0.5cm] % Major heading such as course name
%\textsc{\large Minor Heading}\\[0.5cm] % Minor heading such as course title

%----------------------------------------------------------------------------------------
%	TITLE SECTION
%----------------------------------------------------------------------------------------

\HRule \\[0.4cm]
{ \huge \bfseries Projekt UML Automatu z przek?skami}\\[0.4cm] % Title of your document
\HRule \\[1.5cm]
 
%----------------------------------------------------------------------------------------
%	AUTHOR SECTION
%----------------------------------------------------------------------------------------

\begin{minipage}{0.4\textwidth}
\begin{flushleft} \large 
Autorzy:\\
Marcin \textsc{J?drzejczyk} \\
Sebastian\textsc{ Katszer}\\

\end{flushleft}
\end{minipage}
~
\begin{minipage}{0.4\textwidth}
\begin{flushright} \large
\emph{Prowadz?cy:}\\
 Dr in?. Wojciech \textsc{Szmuc} % Supervisor's Name
\end{flushright}
\end{minipage} \\[4cm]

{\large \today}\\[3cm]
\vfill
\end{titlepage}
%----------------------------------------------------------------------------------------
%	DOCUMENT SECTION
%----------------------------------------------------------------------------------------
\tableofcontents %robi spis tre�ci	- 3 kompilacje
\newpage
\section{Automat z przek?skami }
\subsection{Co to jest?}

\subsection{Analiza wymaga?}

\begin{tabular}{|c|c|c|c|c|} \hline
ID & OPIS & PRIORYTET & KRYTYCZNO?? & HARMONOGRAM \\ \hline
R01 & Wydawanie reszty & Wysoki & ?rednia & Release 1\\ \hline
R02 & Dost?p techniczny dla os�b upowa?nionych & Wysoki & Wysoka & Release 1 \\ \hline
R03 & Spos�b inicjalizacji & ?redni & ?rednia & Release 1\\ \hline
R04 & Wydawanie produktu & ?redni & ?rednia & Release 1 \\ \hline
\end{tabular}

\section{Diagram przypadk�w u?ycia}
\begin{figure}[ht]
\centerline{\includegraphics[scale=0.5]{r2}}
\caption{Diagram sekwencji}
\end{figure}
\subsection{Diagram sekwencji}

\begin{figure}[ht]
\centerline{\includegraphics[scale=0.5]{r2}}
\caption{Diagram sekwencji}
\end{figure}

%\bibliographystyle{unsrt}
%\bibliography{bib2}
\end{document}
